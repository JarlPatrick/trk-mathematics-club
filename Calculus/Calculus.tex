\documentclass[a4paper,11pt,twocolumn]{article}
\usepackage{polyglossia} % Eesti keele tugi
\setdefaultlanguage{estonian}
\usepackage{geometry}% Paigutus
\usepackage{graphicx}% Joonised
\graphicspath{ {images/} }
\usepackage{csquotes}% Eesti jutumärgid \enquote{}
\usepackage{enumitem}% Listid
\usepackage[compact]{titlesec}% Kompaktsed pealkirjad
\usepackage{siunitx}% SI ühikud
\usepackage{tikz}
\usepackage[siunitx]{circuitikz}
\usepackage[final]{microtype}
\usepackage{amsmath}
\usepackage{lmodern}
\geometry{
    paper=a4paper, % Paper size, change to letterpaper for US letter size
    top=0.5cm, % Top margin
    bottom=1cm, % Bottom margin
    left=0.5cm, % Left margin
    right=0.5cm, % Right margin
    foot=0.5cm, % Footer-margin distance
    %showframe, % Uncomment to show how the type block is set on the page
}


%\setlength{\intextsep}{0pt}

\setlength{\parindent}{0cm}% Taandridu pole
\setlength{\parskip}{1em}% Paragraafide vahed
\setlist[itemize]{topsep=0em, partopsep=0em, parsep=0em, itemsep=0.5em}% Itemize spacing

\usepackage{xparse}

% Ülesanded \begin{question}[viide][joonis][joonise suurus] (võib olla ka ainult [viide] või [joonis][joonise suurus])
\newcounter{myproblems}
\NewDocumentEnvironment{question}{o o o}
{\par \refstepcounter{myproblems} \textbf{Ülesanne \themyproblems .} \ignorespaces\IfValueT{#1}{\IfValueTF{#3}{\textbf{(#1)} \ignorespaces}{\IfNoValueT{#2}{\textbf{(#1)} \ignorespaces}}}}
{\IfValueT{#2}{\IfValueTF{#3}{\begin{figure}[h!]\includegraphics[width=#3]{#2.pdf}\centering\vspace{-1em}\end{figure}}{\begin{figure}[h!]\includegraphics[width=#2]{#1.pdf}\centering\vspace{-1em}\end{figure}}}\ignorespacesafterend}

% Alaülesanded
\newenvironment{subquestion}
{\setlength{\parskip}{0pt}\begin{enumerate}[label=\alph*), nolistsep]}
{\end{enumerate}\setlength{\parskip}{1em}\ignorespacesafterend}

% Vihjete jaoks
\newenvironment{hint}[1][Vihje]
{\setlength{\parskip}{0em} \textit{#1}: \ignorespaces}
{\setlength{\parskip}{1em}\ignorespacesafterend}

\newcommand{\pvec}[1]{\vec{#1}\mkern2mu\vphantom{#1}}% Primed vector

% Lahendused jaoks
\usepackage{hyperref}
\newenvironment{solutions}
{\begin{enumerate}[label=\textbf{\arabic*.}, wide]}
{\end{enumerate}}

% Displaystyle valemite paigutus
\makeatletter
\g@addto@macro{\normalsize}{%
    \setlength{\abovedisplayskip}{4pt}
    \setlength{\abovedisplayshortskip}{4pt}
    \setlength{\belowdisplayskip}{4pt}
    \setlength{\belowdisplayshortskip}{4pt}
    }
\makeatother

% \directlua{dofile("DetectUnderfull.lua")}
\tikzset{
    odot/.style={
        circle,
        inner sep=0pt,
        node contents={$\odot$},
        scale=1
    },
    otimes/.style={
        circle,
        inner sep=0pt,
        node contents={$\otimes$},
        scale=1
    }}

\usepackage{blindtext}

\begin{document}
{\huge \textbf{Calculus}\hfill \normalsize {nr 1}} \\
{Jarl Patrick Paide \hfill 14. jaanuar 2020}

\section{Tuletis}

Tuletis näitab funktsiooni muutumise kiirust. Graakifult vaadates on tuletis mingi funktsiooni puutuja tõus mingis punktis. Tuletist võttes saame me uus funktsiooni mis näitab seda tõusu sõltuvuses $x$'ist. Kui meil on funktsioon $f(x)$ siis on selle funktsiooni tuletis $f'(x)$ on deffineeritud järgnevalt
\begin{equation*}
f'(x) = \lim_{\Delta x \to 0} \frac{\Delta y}{\Delta x} = \lim_{\Delta x \to 0} \frac{f(x+\Delta x) - f(x)}{\Delta x}
\end{equation*}
$\Delta x$ asemel võib kasutada tähist $dx$, kui muutus on väga väike
\begin{equation*}
\lim_{\Delta x \to 0} \Delta x = dx
\end{equation*}

Seega kui me võtame tuletis funktsioonist $f(x)$, $x$'i järgi siis saab seda tähistada järgnevalt
\begin{equation*}
f'(x)=\frac{d}{dx}f(x) = \frac{f(x+ dx) - f(x)}{dx}
\end{equation*}

Vaatame, kuidas me saame kasutades tuletis deffinitsiooni leida lihtsa tuletise funktsioonist $f(x) = x$

\begin{equation*}
f'(x) = \frac{(x+dx) - (x)}{dx} = \frac{dx}{dx} = 1
\end{equation*}

Kui mõelda selle funktsiooni graafikule siis on see vastus usutav ja loogiline.

\subsection{Konstandid}
Kui meil on funktsioon $f(x) = a$ siis on see funktsioon konstantne. Graafikul vaadates on tegemist x-teljega paralleelse sirgega, mis läbib y-telge punktis a. Kuna selle funktsiooni tõus igas punnktis on 0, siis on selle funktsiooni tuletis $f'(x) = 0$

Kui meil on funktsioon $f(x) = ag(x)$ siis kasutades tuletise deffinitsiooni saame teha järgnevat
\begin{equation*}
f'(x) = \frac{ag(x+dx)-ag(x)}{dx} = a\frac{g(x+dx)-g(x)}{dx}=ag'(x)
\end{equation*}
Ehk siis kui meil funktsioon $f(x) = ag(x)$ siis me saame tuua konstanti ette ja võtta tuletist ülejäänud funktsioonist.

\subsection{Polünomid}
Vaatame lihtsat funktsiooni $f(x) = x^2$ ja võttame sellest funktsioonist tuletist tuletise deffinitsiooni abiga
\begin{multline*}
f'(x) = \frac{(x+dx)^2-(x)^2}{dx} = \frac{x^2 + 2xdx + dx^2 - x^2}{dx} = \\
= 2x + dx = 2x
\end{multline*}

Kui $x=0$ siis see funktsioon fuutub x-telge, seega on loogiline, et tuletis, ehk tõus, on selles punktsi 0. Kui $x=1$ siis on funktsiooni väärtus 1, aga tõus selles punktis on 2. See funktsioon kahaneb vahemikus $x<0$ seega on loogiline, et seal on funktsiooni tuletis negatiivne. Vastupidi jällegi juhul kui $x>0$. Ning mida suuremaks x muutub seda kiiremini hakkab funktsioon muutuma, seega funktsiooni puutuja tõus suureneb.

Vaatame nüüd polünomi üldjuhtu $f(x) = x^n$
\begin{multline*}
f'(x) = \frac{(x+dx)^n-x^n}{dx} = \frac{(x^n+nx^{n-1}dx+...+dx^n)-x^n}{dx}= \\
=nx^{n-1}+(n(n-1)/2)x^{n-2}dx+...+dx^{n-1}= nx^{n-1}
\end{multline*}
Seega $\frac{d}{dx}(x^n)=nx^{n-1}$, ehk kui $f(x)=2x^3$ siis $f'(x)=6x^2$.

\subsection{Funktsioonide liitmine}
Olgu meil funktsioon $f(x) = g(x) + h(x)$ ja vaatame, kuidas leida tuletist $f'(x)$
\begin{multline*}
f'(x) = \frac{(g(x+dx)+h(x+dx))-(g(x)+h(x))}{dx} = \\
= \frac{g(x+dx) - g(x)}{dx} + \frac{h(x+dx) - h(x)}{dx} = g'(x) + h'(x)
\end{multline*}
Seega kui funktsioon on funktsioonide summa siis on selle funktsiooni tuletis nende summa liikmete tulemiste summa.

\subsection{Funkstioonide korrutamine}
Olgu meil funktsioon $f(x) = g(x)h(x)$ ja vaatame, kuidas leida tuletist $f'(x)$
\begin{equation*}
f'(x) = \frac{(g(x+dx)h(x+dx)) - (g(x)h(x))}{dx} = 
\end{equation*}
Me võime liita ja lahutada murru lugejast sama arvu $g(x)h(x+dx)$ ja me teame, et suvaline funktsioon $k(x+dx) = k(x)$ seega saame
\begin{multline*}
= \frac{g(x+dx)h(x+dx) - g(x)h(x+dx) + g(x)h(x+dx) - g(x)h(x)}{dx} = \\
= \frac{h(x+dx)(g(x+dx)-g(x))}{dx} + \frac{g(x)(h(x+dx)-h(x))}{dx} = \\
= g'(x)h(x) + g(x)h'(x)
\end{multline*}
Seeda saab lihtsamalt kirjutada. Kui meil on funktsioonid $u$ ja $v$, mis mõlemad sõltuvad $x$'ist, siis $(uv)' = u'v + uv'$.

\subsection{Liitfunktsioonid ?}
Olgu meil funktsioon $f(x) = g(h(x))$. Näiteks, seega kui $h(x) = x+1$ ja $g(x) = x^2$ siis $f(x) = g(x+1) = (x+1)^2$. Nüüd vaatame, kuidas sellisest funktsioonist tuletist võtta. Tuletise deffinitsioonist saame
\begin{equation*}
f'(x) = \frac{g(h(x+dx)) - g(h(x))}{h(x+dx) - h(x)} \cdot \frac{h(x+dx) - h(x)}{dx}
\end{equation*}
Kui me teeme asenduse $h(x) = a$ ja $h(x+dx) = a + da$, siis saame, et
\begin{multline*}
f'(x) = \frac{g(a+da) - g(a)}{da} \cdot \frac{h(x+dx) - h(x)}{dx} = \\
= \frac{dg(a)}{da}\cdot\frac{da}{dx} = \frac{dg(h(x))}{dh(x)}\cdot\frac{dh(x)}{dx}
\end{multline*}
Kuna sellest võib olla raske aru saada teeme läbi ühe näite. Olgu meil funktsioon $f(x) = (x+1)^2$, ja võtame $h(x) = x+1$ ja $g(x) = x^2$ siis $f(x) = g(h(x))$. Me võime teha asenduse $a = h(x)$. Seega tuletis on 
\begin{multline*}
f'(x) = \frac{d(x+1)^2}{d(x+1)} \cdot \frac{d(x+1)}{dx} = \frac{d a^2}{da} \cdot \frac{d(x+1)}{dx} =\\
= (2a) \cdot (1+0) = 2(x+1) = 2x+2
\end{multline*}
Kuna me oskame nüüd ka teist moodi seda tuletist võtta teeme kontrolliks ka selle läbi
\begin{equation*}
((x+1)^2)' = (x^2 + 2x + 1)' = (x^2)' + (2x)' + (1)' = 2x + 2
\end{equation*}
Seega kõik töötab

\subsection{Funktsioonide jagamine}
Vaatame nüüd jagatist $\frac{u}{v}$ ja leiame selle tuletise.
\begin{equation*}
(\frac{u}{v})'= (u\cdot\frac{1}{v})'=u'(\frac{1}{v})+u(\frac{1}{v})'
\end{equation*}
Kuna $\frac{1}{v}$ on liitfunktsoon siis selle tuletis on
\begin{equation*}
(\frac{1}{v})' = \frac{d (1/v)}{dv} \cdot \frac{dv}{dx} = \frac{-1}{v^2} \cdot v'
\end{equation*}
Seega on jagatise tuletis 
\begin{equation*}
(\frac{u}{v})'=u'(\frac{1}{v})+u(\frac{1}{v})' = \frac{u'v-uv'}{v^2}
\end{equation*}

\subsection{Sin ja Cos}
Siinuse ja Cosiinuse summa abivalemid
\begin{equation*}
\sin(\alpha\pm\beta) = \sin(\alpha)\cos(\beta)\pm\sin(\beta)\cos(\alpha)
\end{equation*}
\begin{equation*}
\cos(\alpha\pm\beta) = \cos(\alpha)\cos(\beta)\mp\sin(\alpha)\sin(\beta)
\end{equation*}
Võtame funktsioonist $f(x) = \sin(x)$ tuletist kasutades tuletise deffinitsiooni
\begin{equation*}
f'(x) = \frac{\sin(x+dx)-\sin(x)}{dx} =
\end{equation*}
Kuna $dx$ läheneb nullile siis $\cos(dx) = 1$ ja $\sin(dx)/dx=1$ ja sellest saame, et 
\begin{equation*}
= \frac{\sin(x)\cos(dx)+\cos(x)\sin(dx)-\sin(x)}{dx} = \cos(x)
\end{equation*}

Vaatame nüüd juhtu $f(x) = \cos(x)$
\begin{multline*}
f'(x) = \frac{\cos(x+dx)-\cos(x)}{dx} = \\
= \frac{\cos(x)\cos(dx) - \sin(x)\sin(dx) - \cos(x)}{dx} = -\sin(x)
\end{multline*}
Kuna me saame konstantid tuletiest välja võtta saame, et $(-\sin(x))' = -\cos(x)$ ja $(-\cos(x))' = \sin(x)$. Kuna ülejäänud trigonomeetrilisi võrrandeid saab nende kahe abil leida, siis piisab nende kahe tuletisest. Leiame näiteks $\tan(x)$ tuletise
\begin{multline*}
\tan(x)'=(\frac{\sin(x)}{\cos(x)})'  = \frac{\sin(x)'\cos(x)-\cos(x)'\sin(x)}{\cos^2(x)}=\\
=\frac{\cos^2(x) + \sin^2(x)}{\cos^2(x)} = \sec^2(x)
\end{multline*}

\subsection{Logarütm}
Nüüd vaatame logarütmilisi funktsioone. Alustame funktsioonist $f(x) = \ln x$. Selle jaoks meil vaja $e$ deffinitsiooni, mis on järgnev
\begin{equation*}
e=\lim_{n \to \infty}(1+\frac{1}{n})^n \approx 2.718281828459...
\end{equation*}
Kasutades tuletise deffinitsiooni saame järgneva
\begin{multline*}
  f'(x)=\frac{\ln(x+dx) - \ln(x)}{dx}=\frac{1}{dx}\ln(1-\frac{dx}{x})=\\
  =\ln((1+\frac{dx}{x})^{\frac{1}{dx}}
\end{multline*}
Teeme nüüd asenduse $u=\frac{x}{dx}$
\begin{equation*}
\ln((1+\frac{dx}{x})^{\frac{1}{dx}})=\lim_{u \to \infty}\ln((1+\frac{1}{u})^{\frac{u}{x}}) = \lim_{u \to \infty}\frac{1}{x}\ln((1+\frac{1}{u})^{u})=
\end{equation*}
Märkame, et viimase logarütmi sees on $e$, seega jätkame
\begin{equation*}
=\frac{\ln(e)}{x}=\frac{1}{x}
\end{equation*}
Nüüd leiame tuletise suvalisele logarütmile $f(x)=\log_{a}(x)$
\begin{equation*}
f'(x)=(\log_{a}(x))'=(\frac{\ln(x)}{\ln(a)})'=
\end{equation*}
Kuna aga $\ln(a)$ on konstant, võib selle välja tuua ja siis saame
\begin{equation*}
=\frac{1}{\ln(a)}(\ln(x))'=\frac{1}{\ln(a)x}
\end{equation*}

\subsection{eksponendid}
Nüüd vaatame eksponentfunktsioone. Alustame funktsiooniga $f(x)=e^x$. Liitfunktsiooni tuletisest me teame järgnevat
\begin{multline*}
  (\ln(e^x))'=\frac{1}{e^x}(e^x)'\\
  (x)'e^x=(e^x)'\\
  (e^x)'=e^x
\end{multline*}
Vaatame nüüd funktsiooni $f(x)=a^x$. Me saame seda funktsiooni ümber kirjutada kujul $f(x)=e^{x\ln(a)}$.
\begin{multline*}
f'(x)=(e^{x\ln(a)} )'=e^{x\ln(a)}(x\ln(a))'=a^x\ln(a)
\end{multline*}

\subsection{Kokkuvõte}
Nüüd me oleme tõestanud ja näidanud, kuidas võtta elementaarfunktsioonidest tuletise ja kuidas võtta mitmset erinevast funktioonist koosnevast fuktsioonist tuletist. Seega nüüd on oskus võtta kõigist funktsioonidest tuletist. Et võtta kõik kokku on järgnev kokkuvõttev tabel. ($u$ ja $v$ on tabelis x sõltuvad erinevad funktsioonid)

\begin{table}[h]
  \centering
  \caption{Põhifunktsioonide tuletised}
  \begin{tabular}{c | c}
    \hline
    funktsioon & tuletis\\
    \hline
    $(cu)'$ & $cu'$\\
    $(u+v)'$ & $u'+v'$\\
    $(uv)'$ & $u'v+v'u$\\
    $(u/v)'$ & $(u'v-v'u)/v^2$\\
    $du(v)/dx$ & $(du/dv)(dv/dx)$\\
    \hline
    $c$ & $0$\\
    $x^n$ & $nx^{n-1}$ \\
    $\sin x$ & $\cos x$\\
    $\cos x$ & $-\sin x$\\
    $\tan x$ & $\sec^2 x$\\
    $\ln x$ & $1/x$\\
    $\log_a x$ & $1/(x\ln a)$\\
    $e^x$ & $e^x$\\
    $a^x$ & $e^x\ln a$    
  \end{tabular}
\end{table}

\section{Tuletise rakendused ja ülesanded}

\end{document}
