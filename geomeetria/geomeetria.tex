\documentclass[a4paper,11pt,twocolumn]{article}
\usepackage{polyglossia} % Eesti keele tugi
\setdefaultlanguage{estonian}
\usepackage{geometry}% Paigutus
\usepackage{graphicx}% Joonised
\graphicspath{ {images/} }
\usepackage{csquotes}% Eesti jutumärgid \enquote{}
\usepackage{enumitem}% Listid
\usepackage[compact]{titlesec}% Kompaktsed pealkirjad
\usepackage{siunitx}% SI ühikud
\usepackage{tikz}
\usepackage[siunitx]{circuitikz}
\usepackage[final]{microtype}
\usepackage{amsmath}
\usepackage{lmodern}
\geometry{
    paper=a4paper, % Paper size, change to letterpaper for US letter size
    top=0.5cm, % Top margin
    bottom=1cm, % Bottom margin
    left=0.5cm, % Left margin
    right=0.5cm, % Right margin
    foot=0.5cm, % Footer-margin distance
    %showframe, % Uncomment to show how the type block is set on the page
}


%\setlength{\intextsep}{0pt}

\setlength{\parindent}{0cm}% Taandridu pole
\setlength{\parskip}{1em}% Paragraafide vahed
\setlist[itemize]{topsep=0em, partopsep=0em, parsep=0em, itemsep=0.5em}% Itemize spacing

\usepackage{xparse}

% Ülesanded \begin{question}[viide][joonis][joonise suurus] (võib olla ka ainult [viide] või [joonis][joonise suurus])
\newcounter{myproblems}
\NewDocumentEnvironment{question}{o o o}
{\par \refstepcounter{myproblems} \textbf{Ülesanne \themyproblems .} \ignorespaces\IfValueT{#1}{\IfValueTF{#3}{\textbf{(#1)} \ignorespaces}{\IfNoValueT{#2}{\textbf{(#1)} \ignorespaces}}}}
{\IfValueT{#2}{\IfValueTF{#3}{\begin{figure}[h!]\includegraphics[width=#3]{#2.pdf}\centering\vspace{-1em}\end{figure}}{\begin{figure}[h!]\includegraphics[width=#2]{#1.pdf}\centering\vspace{-1em}\end{figure}}}\ignorespacesafterend}

% Alaülesanded
\newenvironment{subquestion}
{\setlength{\parskip}{0pt}\begin{enumerate}[label=\alph*), nolistsep]}
{\end{enumerate}\setlength{\parskip}{1em}\ignorespacesafterend}

% Vihjete jaoks
\newenvironment{hint}[1][Vihje]
{\setlength{\parskip}{0em} \textit{#1}: \ignorespaces}
{\setlength{\parskip}{1em}\ignorespacesafterend}

\newcommand{\pvec}[1]{\vec{#1}\mkern2mu\vphantom{#1}}% Primed vector

% Lahendused jaoks
\usepackage{hyperref}
\newenvironment{solutions}
{\begin{enumerate}[label=\textbf{\arabic*.}, wide]}
{\end{enumerate}}

% Displaystyle valemite paigutus
\makeatletter
\g@addto@macro{\normalsize}{%
    \setlength{\abovedisplayskip}{4pt}
    \setlength{\abovedisplayshortskip}{4pt}
    \setlength{\belowdisplayskip}{4pt}
    \setlength{\belowdisplayshortskip}{4pt}
    }
\makeatother

% \directlua{dofile("DetectUnderfull.lua")}
\tikzset{
    odot/.style={
        circle,
        inner sep=0pt,
        node contents={$\odot$},
        scale=1
    },
    otimes/.style={
        circle,
        inner sep=0pt,
        node contents={$\otimes$},
        scale=1
    }}


\begin{document}
{\huge \textbf{Geomeetria mataklubile}\hfill \normalsize {nr 1}} \\
{Jarl Patrick Paide \hfill 13. jaanuar 2020}

\section{Piirdenurk}

Ringjoone kaarele toetuv kesknurk on kaks korda suurem samale kaarele toetuvast piirdenurgast ja kaks samale kaarele toetuvat piirdenurka on võrdsed.

\begin{question}
	Tõesta eelolev väide ja Thalese teoreem (Iga kolmnurk, mille küljeks on ümberringjoone diameeter on täisnurkne).
\end{question}

\section{Kõõlnelinurk}

Kõõlnelinurk on nelinurk, mille kõik küljed on sama ringjoone kõõlud

\begin{question}
	Tõesta, et kõõlnelinurga vastasnurkade summa on $180^{\circ}$.
\end{question}

\begin{question}
	Olgu ABC teravnurkne kolmnurk, mille kõrguste aluspunktid külgedel BC, CA, AB on vastavalt D, E ja F ning mille kõrguste lõikepunkt on H.
	\begin{subquestion}
		\item Leia jooniselt kuus kõõlnelinurka.
		\item Tõesta, et H on kolmnurga DEF siseringjoone keskpunkt.
		\item Tõesta, et $\triangle AEF$, $\triangle BFD$ ja $\triangle CDE$ on kõik sarnased kolmnurgaga $\triangle ABC$.
		\item Tõesta, et punkti H peegeldus üle külje BC asub kolmnurga ABC ümberringjoonel.
		\item Olgu külje BC keskpunkt M. Tõesta, et kui H peegeldus üle punkti M on H’, siis H’ asub kolmnurga ABC ümberringjoonel.
		\item Tõesta, et AH’ on kolmnurga ABC ümberringjoone diameeter.
		\item Olgu AH keskpunkt N ning (ABC) keskpunkt O. Tõesta, et ANMO on rööpkülik.
		\item Tõesta, et ME ja MF on mõlemad (AEHF) puutujad ning sirge, mis läbib punkti A ning on paralleelne küljega BC on sammuti (AEHF) puutuja.
	\end{subquestion}
\end{question}	

\begin{question}
	Olgu ABCDE kumer viisnurk nii, et BCDE on ruut keskpunktiga O ning $\angle=90^\circ$. Tüesta, et AO poolitab nurka $\angle BAE$.
\end{question}

\begin{question}
	(EGMO 2012) Olgu ABC kolmnurk ning O selle ümberringjoone keskpunkt. Punktid D, E ja F asuvad vastavalt lõikudel BC, CA, AB nii, et $DE\perp CO$ ja $DF \perp BO$. Olgu K kolmnurga AFE ümberringjoone keskpunkt. Tõesta, et $DK \perp BC$.
\end{question}

\begin{question}
	(Venemaa 1996) Kumera nelinurga ABCD küljel BC valitakse punktid E ja F (E asub B ja F vahel). On teada, et $\angle BAE = \angle CDF$ ja $\angle EAF = \angle FDE$. Tõesta, et $\angle FAC = \angle EDB$.
\end{question}

\section{Puutuja-kõõlu teoreem}

Nurk puutuja ja kõõlu vahel on võrdne sellele kõõlule toetuva piirdenurgaga.

\begin{question}
	Tõesta puutuja-kõõlu teoreem
\end{question}

\begin{question}
	(PAMO 2018) Antud on kolmnurk ABC. Küljega AB risti olev sirge läbi tipu A ja küljega BC risti olev sirge läbi tipu B lõikuvad punktis D. Olgu P punkt kolmnurga sisepiirkonnas. Näita, et DAPB on kõõlnelinurk siis ja ainult siis, kui $\angle BAP = \angle CBP$.
\end{question}

\begin{question}
	Kaks ringjoont lõikuvad punktides P ja Q. Läbi punkti A, mis asub esimesel ringjoonel, tõmmatakse sirged AP ja AQ. Need sirged lõikavad teist ringjoont vastavalt punktides B ja C (mis ei kattu P ja Q'ga). Tõesta, et läbi A esimesele ringjoonele tõmmatud puutuja on paralleelne sirgega BC. 
\end{question}

\begin{question}
	Ringjooned $S_1$ ja $S_2$ lõikuvad punktides A ja P. $S_2$ peal valitakse punkt B nii, et AB on $S_1$ puutuja. Sirge CD, kus C asub ringjoonel $S_2$ ja D ringjoonel $S_1$, läbib punkti P ja on paralleelne AB'ga. Tõesta, et ABCD on rööpkülik.
\end{question}

\begin{question}
	(PAMO 2016) Kaks ringjoont $\varphi_1$ ja $\varphi_2$ lõikuvad erinevates punktides M ja N. Nende kahe ringjoone punktide N lähemal olev ühine puutuja puutub ringjoont $\varphi_1$ punktis P ja $\varphi_2$ punktis Q. Sirge PN lõikub $\varphi_2$'ga teist korda punktis R. Tõesta, MQ on nurga $\angle PMR$ poolitaja.
\end{question}

\section{Punkti potents}

Olgu tasandil antud punkt P ja ringjoon c. Läbi punkti P tõmmatakse sirge, mis lõikab ringjoont punktides A ja B (juhul kui sirge on puutuja ühtiva punktid A ja B). Suurust $|PA|\cdot|PB|$ nimetatakse punkti potentsiks ringjoone c suhtes.

\begin{question}
	Tõesta, et punkti P potents ringjoone c suhtes ei sõltu valitud sirgest.
\end{question}

\begin{question}
	Olgu ringjoone c raadius R ja punkti P kaugus ringjoone keskpunktist d. Tõesta, et punkti P potents ringjoone c suhtes on
	\begin{subquestion}
		\item $R^2 - d^2$, kui P asub ringjoone sees;
		\item 0, kui P asub ringjoonel;
		\item $d^2 - R^2$, kui P asub ringjoonest väljaspool.
	\end{subquestion}
\end{question}

\begin{question}
	Tõesta Pythagorase teoreem punkti potentsi abil.
\end{question}

\begin{question}
	(BT 2013 treeningvõistlus) Olgu $\Gamma_1$ ringjoon keskpunktiga O. Ringjoon $\Gamma_2$ läbib punkti O ning lõikab ringjoont $\Gamma_1$ punktides A ja B. Olgu L suvaline O'st erinev punkt ringjoonel $\Gamma_2$. Sirged OL ja AB lõikuvad punktis M. Tõesta, et $|OM|\cdot|OL|=|OA|^2$.
\end{question}

\begin{question}
	(BT 2008) Olgu ABCD rööpkülik. Ringjoon diameetriga AC lõikab sirget BD punktides P ja Q. Punkti C läbiv sirgega AC ristuv sirge lõikab sirgeid AB ja AD vastavalt punktides X ja Y. Tõesta, et punktid P, Q, X ja Y sauvad ühel ringjoonel.
\end{question}

\begin{question}
	(IMO teine valikvõistuls 2018) Olgu AD teravnurkse kolmnurga ABC kõrgus. Sirgel AD valitakse erinevad punktid E ja F nii, et $|DE| = |DF|$ ja punkt E on kolmnurga ABC sisepiirkonnas. Kolmnurga BEF ümberringjoon lõikab lõike BC ja BA teist korda vastavalt punktides K ja M. Kolmnurga CEF ümberringjoon lõikab lõike CB ja CA teist korda vastavalt punktides L ja N. Tõesta, et sirged AD, KM ja LN lõikuvad ühes punktis.
\end{question}

\section{Radikaaltelg ja radikaalkese}

Olgu antud kaks mittekotsentrilist ringjoont $c_1$ ja $c_2$. Punktid, mille potents on mõlema ringjoone suhtes võrdne, moodustavad tasandil sirge. Seda sirget nimetatakse $c_1$ ja $c_2$ radikaalteljeks.

\begin{question}
	Tõesta, et radikaaltelg on hästi defineeritud: punktid, mille potents on mõlema ringjoone suhtes võrdne, asuvad tõepoolest kõik ühel sirgel.
\end{question}

\begin{question}
	Tõesta, et lõikuvate ringjoonte radikaaltelg läbib nende ringjoonte lõikepunkte. 
\end{question}

\begin{question}
	Tõesta, et kolme ringjoone, mille keskpunkt ei asu ühel sirgel, paarikaupa radikaalteljed lõikuvad kõik ühes punktis. (See punkt on nende kolme ringjooone radikaalkese.) Mis juhtub siis, kui ringjoonte keskpunktid asuvad ühel sirgel.
\end{question}

\begin{question}
	Olgu ABC komnurk ning P punkt selle sisepiirkonnas. Oletame, et BC on kolmnurkade ABP ja ACP ümberringjoonte puutuja. Tõest, et AP poolitab BC.
\end{question}

\begin{question}
	Tõesta radikaaltegede kaudu, et kolmnurga kõrgused lõikuvad ühes punktis.
\end{question}

\begin{question}
	(BT 2005) Sirged e ja f ristuvad punktis H. Punktid A ja B asuvad ühel sirgel e ning  punktid C ja D asuvad sirgel f, kusjuures punktid A, B, C, D ja H on paarikaupa erinevad. Sirged b ja d läbivad vastavalt punkte B ja D ning on risti sirgega AC. Sirged a ja d  läbivad vastavalt punkte A ja C ning on risti sirgega BD. Sirgete a ja b lõikepunkt on X ning c ja d lõikepunkt on Y. Tõesta, et sirge XY läbib punkti H.
\end{question}

\begin{question}
	Nelinurga ABCD külgede AB ja CD pikendused lõikuvad punktid F ning külgede BC ja AD pikendused punktis R. Tõesta, et ringjooned diameetritega AC, BD ja EF omavad ühist radikaaltelge ja et kolmnurkade ABE, CDE, ADF ja BCF ümberringjoonte keskpunktid asuvad sellel radikaalteljel.
\end{question}



\end{document}